\documentclass{article}
\usepackage{amsmath}
\usepackage{graphicx}
\usepackage{hyperref}
\usepackage{xcolor}
\title{\textbf{THE 18.821 MATHEMATICS PROJECT LAB REPORT\newline
[REPLACE THIS WITH YOUR OWN SHORT DESCRIPTIVE TITLE!]}}
\author{X.BURPS,P.GURPS}

\begin{document}
\maketitle
\begin{abstract}
	Abstract. This is a LATEX template for 18.821, which you can
use for your own reports
  
	  
\end{abstract}
\section{Introduction}
\centering
This brief document shows some examples of the use of LATEX and
indicates some special features of the Math Lab report style. The
\textcolor{blue}{\url{course website}} contains links to several LATEX manuals.\\
         End the introduction by describing the contents of the paper section by section, and which team member(s) wrote each of them. For
instance, Section \textcolor{blue}{\url{6}} discusses referencing, and is written by P. Gurps.
\section{Latex Examples}
Here are some ways of producing mathematical symbols. Some are
pre-defined either in LATEX or in the AMS.package which this document  loads. For instance, sums and integrals, $\sum_{i=1}^{n} 1 =n,  \int^0_n xdx= \frac{n^2}{2}$. We’ve defined a few other symbols at the start of the document, for
instance N, Q, Z, R. You can make marginal notes for yourself or your
co-authors like this:\\
If you want to typeset equations, there are many choices, with or
without numbering:\\
\centering

$\int^0_1 xdx= \frac{1}{2}$\\

or
 
 $\sum_{i=1}^{\infty} 1 =n$\\
or

$1-1+1-\ldots=\frac{1}{2}$\\
\begin{figure}[h]
    \centering
    \includegraphics[width=0.5\textwidth,height=5cm]{projectt2.jpg}
    \caption{My first .pdf figure.}
    \label{fig:first}
\end{figure}

If you want a number for an equation, do it like this:\\
$(1) \lim_{n\to\infty}
\sum_{k=1}^{n} \frac{1}{k^2}=\frac{\pi}{6}.$\\
This can then be referred to as (\textcolor{blue}{\url{1}}), which is much easier than keeping
track of numbers by hand. To group several equations, aligning on the
= sign, do it like this:\\

$x-1+2x_2+3x_3=7$\\
$y=mx+c$\\
$=4x-9.$\\

You can easily embed hyperlinks into the output .pdf document:
\textcolor{blue}{\url{click here}} for example.
\section{Images}
Figure \textcolor{blue}{\url{1}} is an example of a .pdf image put into a floating environment, which means LaTeX will draw it wherever there’s enough space
left in your manuscript. Look at the .tex original to see how to insert
a figure like this.\\

\section{THEOREMS AND SUCH}
		An example of a “conjecture environment” is given below, in Conjecture \textcolor{blue}{\url{4.1}}. Theorems, lemmas, propositions, definitions, and such all
use the same command with the appropriate name changed. In fact,\\

\vspace*{1cm}
\footnotesize{THE.18.821 REPORT}\\
if you look at the top of this .tex file, you can see where we’ve defined
these environments.

\vspace*{0.5cm}
\textbf {conjecture 4.1}(Vaught’s Conjecture). Let T be a countable complete theory. If T has fewer than 20 many countable models (up to
isomorphism), then it has countably many countable models\\
	
\vspace*{0.2cm}
\textbf{Theorem 4.2.} when it rains it pours.\\
proof.well,yes
	\section{Filetypes used by LaTeX}
You will write your text as a .tex file using any text editor (though
WYSIWYG editors are troublesome). Traditionally one then runs
LATEX and obtains a .dvi file, which can be viewed on the screen using a
dvi viewer. To include images, and then prepare the file for printing or
submission, one typically translates the .dvi into either .ps (Postscript)
or .pdf (Adobe PDF).\\
Your report will be submitted as a .pdf document. The  \texttt{pdflatex}
command produces a .pdf file directly from a .tex file. This command
works well with included .pdf files, but does not handle .eps files.\\
An .eps file can be converted to a .pdf file by viewing it andsaving
as a .pdf file, or by \texttt{ps2pdf filename.eps}, which produces
filename.pdf. Under MikTeX with WinEdt, all necessary commands
will appear under “Accessories” in the WinEdt menu.\\
Finally, Matlab can be made to produce .eps files by typing
\texttt{print -deps filename}\\
at the prompt.
\section{Quoting sources}
In your work, keep notes of the literature you’ve used, including
websites. Cite the references you use; failure to do so constitutes plagiarism. Every bibliography item should be referenced somewhere in
the paper. Quote as precisely as possible: [\textcolor{blue}{\url{1}}, pages 76–78] rather than
[\textcolor{blue}{\url{1}}]. [\textcolor{blue}{\url{2}}] was a useful background reference, too\\




\normalsize{APPENDIX}\\
Appendices are useful for putting in code or data\\
\vspace*{0.3cm}
MIT OpenCourseWare
	\textcolor{blue}{\url{http://ocw.mit.edu}}
\vspace*{0.3cm}\\
\bf 18.821 Project Laboratory in Mathematics\\
Spring 2013\\
\vspace*{0.4cm}
For information about citing these materials or our Terms of Use, visit:\textcolor{blue}{\url{http://ocw.mit.edu/terms}}

\end{document}